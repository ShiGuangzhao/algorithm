\input{/home/shigz/local/courease/Microwave_Measure/headfile}
\usepackage[framed,numbered,autolinebreaks,useliterate]{/home/shigz/local/courease//qianrushi/mcode}
\lstset{basicstyle=\footnotesize\ttfamily,breaklines=true, language=C, xleftmargin=1em, xrightmargin=1em,  aboveskip=1em}

\begin{document}
\begin{center}
    {\heiti\zihao{-2}\textbf{《算法导论》笔记}}\\  
\end{center}
\begin{spacing}{1.2}
\songti\zihao{-4}
\section{C语言知识}
\subsection{关于变长数组}
使用变量定义数字元素个数,即

\qquad int str[number];

\subsubsection{用法}
\begin{enumerate} [(1)]
    \item 可在程序块中使用(即不能为全局变量)
    \item 不可作为结构体成员或联合体成
    \item 不能用static或extern修饰
    \item 一定要用的话可以在堆上定义(malloc/free)
\end{enumerate}

\subsection{统计程序运行时间}
\subsubsection{使用clock函数}
\begin{lstlisting}
#include <time.h>
double start, end; // clock_t?
start = clock();
...
end = clock();
double time = end - start;
\end{lstlisting}

\subsection{使用linux命令}
命令如下:

\qquad \$ time ./result

\begin{lstlisting}
    real	0m8.170s    // 实际运行时间
    user	0m0.001s    // 用户空间运行时间
    sys	    0m0.001s    // 内核空间运行时间
\end{lstlisting}

\subsection{C语言产生随机数}
\begin{lstlisting}
#include <stdlib.h>
#include <time.h>
// 以时间为种子,如不调用此函数则每次运行结果相同
// 此函数只调用一次即可 
srand(time(NULL));
// 产生 0-100 的随机数
int number = rand()%101;
\end{lstlisting}

\subsection{C语言计算对数}
math.h 中包含log(以e为底)和log10两个函数,计算其他的可以应用公式
\[
    \log _ab = \frac{\log _cb}{\log _ca}
\]
如
\[
    \log _24 = \frac{\log 4}{\log 2}
\]

    \section{其他知识}
    \subsection{Linux使用数学库}
    使用math.h时编译如下

    \qquad \$ gcc area.c -o area.out  -lm

\end{spacing}
\end{document}

